\documentclass[10pt,a4paper]{article}
\usepackage[utf8]{inputenc}
\usepackage[parfill]{parskip}
\usepackage[section]{placeins}
\usepackage{graphicx}
\usepackage{array}
\usepackage{tabularx}
\usepackage[scientific-notation=true]{siunitx}
\usepackage{amsmath}

\author{Christoph Emunds (i6146758)\\Dominik Nerger (i6146759)}
\title{Autonomous Robotic Systems\\Bayes Filter}
\date{\today}

\begin{document}
	\maketitle
	
	\tableofcontents
	
	\section{Introduction}
	
	\section{The world model}
	
	\section{The Bayes Filter}
	Starts with a uniformly distributed belief state
	
		\subsection{Update step}
		The update step is executed whenever the robot moves forward or makes a 180 degree turn.
		To calculate the probability for being in a state $x$ after a motion command, the probabilities of all possible starting states is weighted according to the given transition probabilities and summed up. For the \textit{move forward} command, this equates to:
		
		\begin{displaymath}
			p(x_i) = 0.8\times p(x_{i-1}) + 0.1\times p(x_{i-2}) + 0.1\times p(x_i)
		\end{displaymath}
		
		Similarly, for the \textit{turn} command, this results in:
		
		\begin{displaymath}
			p(x_i) = 0.9\times p(x_{N-1-i}) + 0.1\times p(x_i)
		\end{displaymath}
		
		\subsection{Measurement step}
		The probability for being in a state that is conform with the measurement is multiplied by a factor of $0.6$, the probability for every other state is multiplied by $0.2$.
	
	\section{Results}
	Most states in the corridor have a wall on both sides, which makes them indistinguishable.
	When issuing the \textit{measurement} command in a state that has a door, the probabilities of the three states containing a door raises. Multiple continuous measurements cause the robot's confidence for being in one of these three states to grow.
	
	\section{Conclusion}
	
\end{document}